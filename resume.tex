%!TEX TS-program = xelatex
%!TEX encoding = UTF-8 Unicode
% Awesome CV LaTeX Template
%
% This template has been downloaded from:
% https://github.com/posquit0/Awesome-CV
%
% Author:
% Claud D. Park <posquit0.bj@gmail.com>
% http://www.posquit0.com
%
% Template license:
% CC BY-SA 4.0 (https://creativecommons.org/licenses/by-sa/4.0/)
%


%%%%%%%%%%%%%%%%%%%%%%%%%%%%%%%%%%%%%%
%     Configuration
%%%%%%%%%%%%%%%%%%%%%%%%%%%%%%%%%%%%%%
%%% Themes: Awesome-CV
\documentclass[]{awesome-cv}
\usepackage{textcomp}
%%% Override a directory location for fonts(default: 'fonts/')
\fontdir[fonts/]

%%% Configure a directory location for sections
\newcommand*{\sectiondir}{resume/}

%%% Override color
% Awesome Colors: awesome-emerald, awesome-skyblue, awesome-red, awesome-pink, awesome-orange
%                 awesome-nephritis, awesome-concrete, awesome-darknight
%% Color for highlight
% Define your custom color if you don't like awesome colors
\colorlet{awesome}{awesome-emerald}
%\definecolor{awesome}{HTML}{CA63A8}
%% Colors for text
%\definecolor{darktext}{HTML}{414141}
%\definecolor{text}{HTML}{414141}
%\definecolor{graytext}{HTML}{414141}
%\definecolor{lighttext}{HTML}{414141}

% personal info
%  Comment any of the lines below if they are not required

% Available options: circle|rectangle,edge/noedge,left/right
% \photo[rectangle,edge,right]{./examples/profile}
\name{Ethan}{Hullett}
%\position{Software Architect{\enskip\cdotp\enskip}Security Expert}
\address{London, UK}

%\mobile{(+44) \ldots}
\mobile{\ldots \ldots}
\email{eh443@kent.ac.uk}
\link{www.ethull.com}
% \github{www.github.com/ethull}
\githublink{www.github.com/ethull}
% \linkedin{ethan-hullett}
% \gitlab{gitlab-id}
% \stackoverflow{SO-id}{SO-name}
% \twitter{@twit}
% \skype{skype-id}
% \reddit{reddit-id}
% \medium{madium-id}
% \googlescholar{googlescholar-id}{name-to-display}
%% \firstname and \lastname will be used
% \googlescholar{googlescholar-id}{}
% \extrainfo{extra informations}

% \quote{``Be the change that you want to see in the world."}


%%%%%%%%%%%%%%%%%%%%%%%%%%%%%%%%%%%%%%
% start doc
%%%%%%%%%%%%%%%%%%%%%%%%%%%%%%%%%%%%%%

%%% Override a separator for social informations in header(default: ' | ')
%\headersocialsep[\quad\textbar\quad]

\begin{document}

    
%%%%%%%%%%%%%%%%%%%%%%%%%%%%%%%%%%%%%%
%     Profile
%%%%%%%%%%%%%%%%%%%%%%%%%%%%%%%%%%%%%%

\makecvheader
\makecvfooter
  {\today}
  {Ethan Hullett~~~·~~~Résumé}
  {\thepage}

%%%%%%%%%%%%%%%%%%%%%%%%%%%%%%%%%%%%%%
%     Education
%%%%%%%%%%%%%%%%%%%%%%%%%%%%%%%%%%%%%%
\cvsection{Education}
\begin{cventries}
    % sage: \cventry{<position>}{<title>}{<location(right)>}{<date(right)>}{<desc>}
    \cventry
    {BSc in Computer Science with Year in Industry}
	{University of Kent}
	{Canterbury}
	{2020 – 2023}
	%{1st year results: distinction \\
    %My favourite module within the first year is the OOP (object oriented programming), which I got a 90\% final mark on
    %}
    %{\def\arraystretch{1.15}{\begin{tabular}{ l l }
	%1st year results: distinction  & \\
    %    My favourite module within the first year was OOP (object oriented programming), which I got a 90\% final mark on. & \\
    %    \end{tabular}}}
    {{\begin{tabular}{p{0.9\textwidth}}
        1st year results: distinction, 85.75\% overall   \\
        \begin{cvitems}
            \item {One of my favourite modules in year 1 was Object Oriented Programming, which I scored 90\% in overall, this module was focused around java OOP conceps, such as polymorphism and encapsulation.}
            \item {Currently one of the modules I am studying is web development, with the theme on MVC frameworks such as CodeIgnitor, which is written in PHP.}
            \item {Later on this year I will study a software development module, which will teach software development concepts such as the agile methodology.}
        \end{cvitems}
    \end{tabular}}
    }

    \cventry
	{A levels}
	{Chis and Sid Grammer School}
	{London}
	{2017 – 2019}
	{1B, 2Cs, including: maths, computer science, business}
	\cventry
	{GCSEs}
	{Hurstmere High School}
	{London}
	{2012 – 2016}
	{8 subjects at A grade, including: maths, science, computer science}
\end{cventries}

\vspace{-2mm}
%%%%%%%%%%%%%%%%%%%%%%%%%%%%%%%%%%%%%%
%     Experience
%%%%%%%%%%%%%%%%%%%%%%%%%%%%%%%%%%%%%%
\cvsection{Experience}
\begin{cventries}
	\cventry
	{CEO assistant}
	{TrainAsOne}
	{Norfolk UK}
	{June 2018 – August 2018}
    {{\begin{tabular}{p{0.9\textwidth}}
        After testing their web app before its release as a beta tester for 2 years, I went to Norfolk for work experience at TrainAsOne. Which is a startup firm that uses AI as a personal trainer for long distance runners. Some skills I developed:   \\
        \begin{cvitems}
            \item {Exposure to apache tomcat (java servlet container)}
            \item {Basic shell script (bash) and automation}
            %\item {Maintenance of PostgreSQL server}
            \item {Brainstorm new features}
            %\item {Clean website Code}
            \item {Test AI product actively (beta testing)}
        \end{cvitems}
    \end{tabular}}
    }

	\cventry
	{IT Solutions Assistant}
	{ABC Computers}
	{London UK}
	{June 2016 – August 2016}
    {{\begin{tabular}{p{0.9\textwidth}}
        I worked unpaid for this computer repair company, in their main office among 4 staff members and the owner. My main responsibilities and skills gained where:   \\
	    \begin{cvitems}
            \item {Commercial Maintenance of client computers (Malware removal, cleaning, defrag)}
            \item {Setup of client computers outside office under client instruction}
            \item {Technical problem solving}
		\end{cvitems}
    \end{tabular}}
    }
\end{cventries}

\cvsection{Computing Skills}
\begin{cvskills}
  \cvskill
    {Programming Languages} % Category
    {Java, Python, Javascript, PHP, Bash} % Skills
  \cvskill
    {Markup Languages} % Category
    {HTML5, CSS, Markdown, SQL} % Skills
  \cvskill
    {Frameworks exposed too} % Category
    {ReactJS, NextJS} % Skills
  \cvskill
    {Data Interchange Formats} % Category
    {YAML, JSON, CSV, XML} % Skills
  \cvskill
    {Linux Tools} % Category
    {GIT, GPG, VIM} % Skills
  \cvskill
    {Operating Systems} % Category
    {Linux system maintenance and troubleshooting (APT, SYSTEMD, GRUB, \ldots)} % Skills
  \cvskill
    {Microsoft} % Category
    {Word processing apps, Sysinternals Tools} % Skills
  %\cvskill
  %  {Other} % Category
  %  {Drivers license} % Skills
\end{cvskills}

%\begin{cventries}
%	\cventry
%	{}
%	{\def\arraystretch{1.15}{\begin{tabular}{ l l }
%		Programming Languages:   & {\skill{Python, Javascript, PHP, Java, Bash}} \\
%		Markup Languages:   & {\skill{HTML5, CSS, Markdown, SQL}} \\
%		Frameworks exposed too:   & {\skill{ReactJS, NextJS}} \\
%        Data Interchange Formats: \hspace{2pt}   & {\skill{YAML, JSON, CSV, XML}} \\
%        Linux Tools:   & {\skill{GIT, GPG, VIM}} \\
%        Operating Systems:   & {\skill{Linux system maintenance and troubleshooting (APT, SYSTEMD, GRUB, \ldots)}} \\
%        Microsoft:   & {\skill{Word processing apps, Sysinternals Tools}} \\
%        Other:   & {\skill{Drivers license}} \\
%		\end{tabular}}}
%	{}{}{}

%\end{cventries}

\vspace{-7mm}
\cvsection{Projects}
\begin{cventries}
	\cventry
	{Personal Website and Blog}
	{ethull}
	{HTML, CSS, JS, Jekyll}
	{https://ethull.com}
	{Personal website built with the static site generator jekyll, has blog posts written in markdown which are converted to html using kramdown.}
	\cventry
	{Summer of Shipping Project}
	{youtube-nlp}
	{HTML, CSS, JS, NextJS, Python, Balsamiq}
	{to be deployed}
	{During the summer I joined the summer of shipping discord and was accepted into the frontend team for the YoutubeNLP project. I wrote the balsamiq mockups for the site and contributed to the nextjs code. Here i learnt to collaborativelly use git with github. When delegated work I would create a feature branch within my fork, incorporate upstream changes, and pull request before the deadline.}
	\cventry
	{configuration for a debian based linux system}
	{dotfiles}
	{Bash, Vimscript, Conf}
	{https://github/ethull/dotfiles}
    {Dotfiles are the hidden files starting with a dot, they are used to configure programs on linux. This repo has configuration files for programs i use so that i can copy my configuration between computers and deploy it quickly. Example programs configured: xterm (terminal emulator), vim (text editor), tmux (terminal multiplexor).}
	
	\vspace{-3mm}
\end{cventries}

%\cvsection{awards}
%\begin{cvhonors}
%    % name,desc,sider,date
%	\cvhonor
%	{drivers license}
%	{}
%	{}
%	{2019}
%\end{cvhonors}

\cvsection{interests}
\begin{cventries}
	\cventry
	{Technology}
	{}
	{}
	{}
    {Enjoy experimenting with software and linux, projects can be found on my github}
	\cventry
	{Fitness/outdoors}
	{}
	{}
	{}
    {Endurance running, mountain biking, gym, DIY.}
    
    
    %\href{http://\@githublink}{\faGithubSquare\ \@githublink}
\end{cventries}

\cvsection{references}
\begin{cventries}
	\cventry
	{ref dude at kent}
	{academic}
	{}
	{}
    {\begin{tabular}{p{0.9\textwidth}}
    email: abc@kent.ac.uk \\ phone: \ldots \ldots
    \end{tabular}}
	\cventry
	{ref dude at trainasone}
	{work experience}
	{}
	{}
    {\begin{tabular}{p{0.9\textwidth}}
        email: abc@trainasone.com \\ phone: \ldots \ldots
    \end{tabular}}
\end{cventries}

\ 
\end{document}
